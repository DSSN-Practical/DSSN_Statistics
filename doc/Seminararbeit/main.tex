\documentclass{article}
\usepackage[ngerman]{babel}
\usepackage[utf8]{inputenc}
\usepackage{tabularx}
\usepackage{listings}
\usepackage{natbib}
\usepackage{graphicx}
\usepackage{url}

\title{Praktikumsarbeit DSSN WS 2014: \\ Statistik der Xodx-Simulation}
\author{Franz Teichmann}
\date{\today}


\begin{document}
\shorthandoff{"}
\maketitle

\tableofcontents
\newpage

%TODO ARBEIT statt Seminararbeit o. Praktikum

\section{Einleitung und Aufgabenstellung}

- im diesjährigen DSSN-Praktikum im Rahmen des Seminares angewandte semantische Technologien sollte im unabhängigen Studententeam
aufgaben bearbeitet werden

- Testen der xodx software und die Arbeit vom Team des letzten Jahres fortzuführen

Die Ergebnisse dieser Arbeit sollen hier zusammenfasst werden


- Semantic Web Vision und Internet 2.0
- soziale Netzwerke als Schlüsselkomponente
- Schwächen und Alternativen

\subsection{Vorarbeit}

- letztes Jahr: Studententeam 5, später 3
- Docker-Krams?
- Twitter Korpus


\section{Begriffe und Technologien}

- folgende Technologien fanden bei der Arbeit Verwendung
- umfangreiche Recherchethemen kurz vorgestellt mit starkem Fokus auf Relevanz für das Projekt

\subsection{Semantic Web}

\subsubsection{Triplestores}

- Apache Jena
- Virtuoso

\subsubsection{Datacube-Vokabular}

- kurz wie Vortrag am Bild

\subsubsection{DSSN / Xodx}

- distributed, social, semantic Network
- Verwendung von Zend, Saft und Erfurt Zusammenhang

\section{Ziele und Planung der Simulation}

- Überprüfung auf Fehlerklassen

\section{Architektur des Statistikkontrollers}

- nach Zend
- public
- private
- Verwendung xodx Model
- Aufruf bei jedem Messzeitpunkt bei jedem Agenten geplant

\section{Auswertung}

- zur Auswertung gehört 
1. Überprüfung des Datacube auf Validität
2. Auswertung nach Gesichtspunkten
2.1. Verlorene Nachrichten

3. Visualisierung
3.1. Möglichkeiten mit CubeViz

\section{Zusammenfassung}

- "Mitglied abgesprungen, kein Test möglich"

\bibliographystyle{plain}
\bibliography{references.bib}
\end{document}
