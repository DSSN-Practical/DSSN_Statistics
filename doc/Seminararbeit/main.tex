\documentclass{article}
\usepackage[ngerman]{babel}
\usepackage[utf8]{inputenc}
\usepackage{tabularx}
\usepackage{listings}
\usepackage{natbib}
\usepackage{graphicx}
\usepackage{url}

\title{Praktikumsarbeit DSSN WS 2014: \\ Statistik der Xodx-Simulation}
\author{Franz Teichmann}
\date{\today}


\begin{document}
\shorthandoff{"}
\maketitle

\tableofcontents
\newpage

%TODO ARBEIT statt Seminararbeit o. Praktikum

\section{Einleitung und Aufgabenstellung}

- im diesjährigen DSSN-Praktikum im Rahmen des Seminares angewandte semantische Technologien sollte im unabhängigen Studententeam
Aufgaben bearbeitet werden
- Simulation zum Test der Xodx- Software
- Testen der xodx software und die Arbeit vom Team des letzten Jahres fortzuführen

Meine Aufgabe war es, 
- Statistikkontroller zu entwerfen und zu implementieren

Die Ergebnisse dieser Arbeit sollen hier zusammenfasst werden

Dazu soll zunächst die im letzten Jahr erfolgte Vorarbeit zusammengefasst werden
dann ein kurzer Überblick über die Recherche zum Thema
- Designübersicht des Statistikkontrollers
- Überblick über den Datenfluss und die Auswertung

alle Dateien mit Dokumentation sowie dieser Arbeit finden sich unter ...



\section{Recherche}

- folgende Technologien fanden bei der Arbeit Verwendung
- umfangreiche Recherchethemen kurz vorgestellt mit starkem Fokus auf Relevanz für das Projekt

\subsection{Semantic Web}

- Semantic Web Vision und Internet 2.0


\subsection{Triplestores}

- Apache Jena
- Virtuoso

\subsection{Datacube-Vokabular}

- kurz wie Vortrag am Bild

\subsection{DSSN / Xodx}

- soziale Netzwerke als Schlüsselkomponente
- Schwächen und Alternativen
- distributed, social, semantic Network
- Verwendung von Zend, Saft und Erfurt Zusammenhang

\section{Vorarbeit}

- letztes Jahr: Studententeam 5, später 3
- Docker-Krams?
- Twitter Korpus
- Planung der Statistikkomponente:
	- Feststellung von drei Fehlerklassen
	- Entwurf eines Datacube
- dieser Datacube wurde geupdatet auf W3C blah


\section{Architektur des Statistikkontrollers}

- nach Zend
- public
- private
- Verwendung xodx Model
- Aufruf bei jedem Messzeitpunkt bei jedem Agenten geplant

\section{Auswertung}

- am Datenflussdiagramm?
- zur Auswertung gehört 
1. Überprüfung des Datacube auf Validität
2. Auswertung nach Gesichtspunkten
2.1. Verlorene Nachrichten

3. Visualisierung
3.1. Möglichkeiten mit CubeViz

\section{Zusammenfassung}

- "Mitglied abgesprungen, kein Test möglich"

\bibliographystyle{plain}
\bibliography{references.bib}
\end{document}
