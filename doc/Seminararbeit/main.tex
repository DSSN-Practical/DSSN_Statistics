\documentclass{article}
\usepackage[ngerman]{babel}
\usepackage[utf8]{inputenc}
\usepackage{tabularx}
\usepackage{listings}
\usepackage{natbib}
\usepackage{graphicx}
\usepackage{url}

\title{Praktikumsarbeit DSSN WS 2014: \\ Statistik der Xodx-Simulation}
\author{Franz Teichmann}
\date{\today}


\begin{document}
\shorthandoff{"}
\maketitle

\tableofcontents
\newpage

%TODO ARBEIT statt Seminararbeit o. Praktikum

\section{Einleitung und Aufgabenstellung}


% im diesjährigen DSSN-Praktikum im Rahmen des Seminares angewandte semantische Technologien sollte im unabhängigen Studententeam
%Aufgaben bearbeitet werden
Das diesjährige DSSN-Praktikum (xodx) war in das Seminar ,,Anwendung Semantischer Technologien'' am Lehrstuhl für betriebliche Informationssysteme eingebettet und umfasste verschiedene praktische Aufgabenkomplexe,
die in einem unabhängigen Studententeam bearbeitet wurden.\\
Die Aufgaben drehten sich darum, einen Test für die bestehende xodx-Software zu entwerfen und eine agentenbasierende Simulationssoftware als Testumgebung nach einem Komponentenmodell zu realisieren. Dabei sollte die Arbeit des Teams vom vorangegangenen Wintersemester fortgeführt werden.\\
Meine Aufgabe war es, die Statistikkomponente als zentrales Verbindungsstück zwischen der xodx-Software bzw. deren Agenten und der Simulationskontrolleinheit zu entwerfen und zu implementieren. Die Ergebnisse dieser Arbeit sollen hier übersichtlich dargestellt werden.\\
Dazu soll zunächst kurz die thematische Recherche zum Thema Anwendung Semantischer Technologien dargestellt werden, um einen Einblick in die Designentscheidungen zu geben. Danach soll die Vorarbeit aus dem letzten Jahr zusammengefasst werden, um anschließend genauer auf den entwickelten Statistikkontroller einzugehen und die Architektur zu umreißen. Anschließend sollen der Arbeitsablauf für die Simulation sowie der anschließende Auswertungsprozess erläutert werden und in der Zusammenfassung eine abschließende Darstellung der erreichten und der noch offenen Zielstellungen gegeben werden.\\
Alle Dateien sowie dieses Dokument und weitere Dokumentation sind auf Github\footnote{\url{https://github.com/DSSN-Practical/DSSN_Statistics}} zu finden.

\section{Recherche}

- folgende Technologien fanden bei der Arbeit Verwendung
- umfangreiche Recherchethemen kurz vorgestellt mit starkem Fokus auf Relevanz für das Projekt

\subsection{Semantic Web}

- Semantic Web Vision und Internet 2.0


\subsection{Triplestores}

- Apache Jena
- Virtuoso

\subsection{Datacube-Vokabular}

- kurz wie Vortrag am Bild

\subsection{DSSN / Xodx}

- soziale Netzwerke als Schlüsselkomponente
- Schwächen und Alternativen
- distributed, social, semantic Network
- Verwendung von Zend, Saft und Erfurt Zusammenhang

\section{Vorarbeit}

- letztes Jahr: Studententeam 5, später 3
- Docker-Krams?
- Twitter Korpus
- Planung der Statistikkomponente:
	- Feststellung von drei Fehlerklassen
	- Entwurf eines Datacube
- dieser Datacube wurde geupdatet auf W3C blah


\section{Architektur des Statistikkontrollers}

- nach Zend
- public
- private
- Verwendung xodx Model
- Aufruf bei jedem Messzeitpunkt bei jedem Agenten geplant

\section{Auswertung}

- am Datenflussdiagramm?
- zur Auswertung gehört 
1. Überprüfung des Datacube auf Validität
2. Auswertung nach Gesichtspunkten
2.1. Verlorene Nachrichten

3. Visualisierung
3.1. Möglichkeiten mit CubeViz

\section{Zusammenfassung}

- "Mitglied abgesprungen, kein Test möglich"

\bibliographystyle{plain}
\bibliography{references.bib}
\end{document}
