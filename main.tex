\documentclass{article}
\usepackage[german]{babel}
\usepackage[utf8]{inputenc}
\usepackage{tabularx}
\usepackage{listings}
\usepackage{natbib}
\usepackage{graphicx}

\title{Praktikumsarbeit DSSN WS 2014: \\ Statistik der Xodx-Simulation}
\author{Franz Teichmann}
\date{\today}


\begin{document}

\maketitle

\tableofcontents
\newpage

\section{Einleitung}

Im diesjährigen DSSN-Praktikum (Xodx) ging es im Allgemeinen um die Simulation eines Verteilten Sozialen Netzwerkes, in welchem semantische Technologien zum Einsatz kommen. Im Team von 5, später 3 Studenten war es unsere Aufgabe, mehrere Instanzen der Xodx-Software auf Docker-Instanzen aufzusetzen und unter Verwendung eines Aktivitätskorpus bestehend aus Twitterdaten diese, nachfolgend Agenten genannten Instanzen, miteinader kommunizieren zu lassen. \\
Es sollten dabei typische Aktivitäten in einem sozialen Netzwek <--break-->



Anschließend sollte die Funktionalität, Korrektheit und Belastbarkeit dieser Software ausgewertet werden. Dabei ging es bewusst nicht darum, Eigenschaften des sozialen Netzwerkes und damit die Qualität oder Repräsentativität der Simulation zu analysieren. Es sollte vielmehr festgestelt werden, ob funktionale Mängel in der Versuchsanordnung bestanden, wenn zum Beispiel Nachrichten nicht ankommen oder die lokalen Speicher der Agenten sich unerwartet verhalten.

\section{Aufgabenstellung und Ziel der Messung}

Ziel: - Funktionalität, (Effizienz)
\begin{enumerate}
    \item{Überprüfung der Dateneffizienz des Xodx-Netzwerkes - Triple-Explosion}
    \item{Überprüfung auf verlorene Nachrichten}
    \item{Zugriffsgeschwindigkeit auf eigene Zeitleiste}
\end{enumerate}


\section{Datacube-Vokabular}

\subsection{Modellierung der Messungen in RDF}

W3C Beispiel

\begin{table}[h]
\begin{tabular}{|l|l|l|l|}
\hline
male           & 2004-2006 & 2005-2007 & 2006-2008 \\ \hline
Newport        & 76.7      & 77.1      & 77.0      \\ \hline
Cardiff        & 78.7      & 78.6      & 78.7      \\ \hline
Monmouthshire  & 76.6      & 76.5      & 76.6      \\ \hline
Merthyr Tydfil & 75.5      & 75.5      & 74.9      \\ \hline
\end{tabular}
\end{table}

\begin{table}[h]
\begin{tabular}{|l|l|l|l|}
\hline
female         & 2004-2006 & 2005-2007 & 2006-2008 \\ \hline
Newport        & 80.7      & 80.9      & 81.5      \\ \hline
Cardiff        & 83.3      & 83.7      & 83.4      \\ \hline
Monmouthshire  & 81.3      & 81.5      & 81.7      \\ \hline
Merthyr Tydfil & 79.1      & 79.4      & 79.6      \\ \hline
\end{tabular}
\end{table}

unser Beispiel

\begin{table}[h]
\begin{tabular}{|l|l|l|l|l}
\cline{1-4}
node           & 1  & 2 & 3   &  \\ \cline{1-4}
size (kByte)   & 10 & 8 & 15  &  \\ \cline{1-4}
follower(number)       & 3  & 8 & 100 &  \\ \cline{1-4}
activity(tweet/day) & 4  & 3 & 15  &  \\ \cline{1-4}
\end{tabular}
\end{table}

\subsection{Implementierung der Messfunktionen}

\section{Darstellung und Auswertung des Datacubes}

\section{Zusammenfassung}
``I always thought something was fundamentally wrong with the universe'' \citep{adams1995hitchhiker}

\bibliographystyle{plain}
\bibliography{references}
\end{document}

